\documentclass[12pt,a4paper]{article}
\newcommand{\version}{4.0}

\usepackage{pdfpages}
\usepackage{url,graphicx,tabularx,array,titleref,booktabs}
\usepackage[hmargin=.47in,vmargin=0.6in,nohead]{geometry}
\usepackage{color,verbatim,framed}
\usepackage{moreverb}
\usepackage{xr}
%\usepackage{upgreek}
\usepackage{fancyvrb}
%\usepackage{fancyhdr}
\usepackage{amsmath}
\usepackage{appendix}
\usepackage{color}
\usepackage{tabularx}
\usepackage{graphicx}
%\usepackage{fullpage}
\usepackage{enumerate}
\usepackage{bold-extra}
%\usepackage{fix-cm}
\definecolor{txt}{rgb}{0,0,1}
\definecolor{cmd}{rgb}{0,1,0}
\newcommand{\txt}[1]{{\color{txt}{\tt #1}}}
\newcommand{\cmd}[1]{{\color{cmd}{\tt #1}}}
\definecolor{warwickdark}{cmyk}{1.0,0.6,0.0,0.06}
\definecolor{warwickmid}{cmyk}{0.69,0.34,0.0,0.0}
\definecolor{warwicklight}{cmyk}{1.0,0.0,0.0,0.0}
\definecolor{warwickred}{cmyk}{0.0,1.0,0.65,0.15}
\newcommand{\HRule}{\rule[0.3cm]{\linewidth}{0.5mm}}
\newcommand{\emphtext}{\color{warwickdark} \fontfamily{phv}\selectfont\large\bf}
\newcommand{\inlinecode}[1]{{\color{warwickred} \bf\texttt{#1}}}
\newcommand{\sect}[1]{Section~\ref{sec:#1}}
\newcommand{\sectit}[1]{``{\bf\titleref{sec:#1}}'' (\sect{#1})}
\newcommand{\code}[1]{{\texttt{#1}}}
%\newcommand{\qtt}[1]{``{\code{#1}}''}
\newcommand{\inlineemph}[1]{{\color{warwicklight} \bf{#1}}}
\newcommand{\inlineemtt}
  {\color{warwicklight}\fontfamily{phv}\selectfont\ttfamily\bfseries}
\newcommand{\cemph}[1]{{\inlineemph{#1}}}
\newcommand{\EPOCH}{{\color{warwickdark}\fontfamily{phv}\selectfont{EPOCH}}}
% Caption before Label to fix strange problem with not putting in subsection
% numbers. DO NOT CHANGE.
\newcommand{\captionedimage}[3]
  {{\begin{figure}[hbt!]\centering\includegraphics{#1}\caption{#3}\label{#2}
    \end{figure}}}
\newcommand{\scaledcapimage}[4]
  {{\begin{figure}[hbt!]\centering\includegraphics[scale=#4]{#1}\caption{#3}
    \label{#2} \end{figure}}}

\newcommand{\tony}[1]{{\color{warwickred} \bf{TONY'S COMMENT:} \bf{#1}}\\}

\definecolor{shadecolor}{cmyk}{0.1,0.05,0.0,0.0}
\setlength{\FrameRule}{0.6mm}

\newenvironment{lboxverbatim}[1]{
\setlength{\FrameSep}{0pt}
%\topsep=0ex\relax
\def\FrameCommand{\fboxsep=0pt \colorbox{shadecolor}}
\MakeFramed{\FrameRestore}
\vspace{-13.5pt}
\fvset{label=#1}
\boxverb
}{
\endboxverb
\vspace{-13.5pt}
\endMakeFramed
}
\newenvironment{boxverbatim}{\lboxverbatim{none}}{\endlboxverbatim}

\newenvironment{lboxverbatim2}[1]{
\setlength{\FrameSep}{0pt}
\topsep=0ex\relax
\def\FrameCommand{\fboxsep=0pt \colorbox{shadecolor}}
\MakeFramed{\FrameRestore}
\vspace{-13.5pt}
\fvset{label=#1}
\boxverb
}{
\endboxverb
\vspace{-13.5pt}
\endMakeFramed
}

\newenvironment{nbboxverbatim}[1]{
\noindent\minipage{\textwidth}
\setlength{\FrameSep}{0pt}
%\topsep=0ex\relax
\def\FrameCommand{\fboxsep=0pt \colorbox{shadecolor}}
\MakeFramed{\FrameRestore}
\fvset{label=#1}
\boxverb
}{
\endboxverb
\vspace{-13.5pt}
\endMakeFramed
\endminipage
\vspace{5pt}
}

\DefineVerbatimEnvironment{boxverb}{Verbatim}
  {frame=single,framerule=0.5mm,rulecolor=\color{warwickmid},
   formatcom=\color{black}}

\DefineVerbatimEnvironment{codedef}{Verbatim}
  {formatcom=\color{warwickred},fontsize=\Large,commandchars=\\\{\}}
%\newcommand{\txt}[1]{{\color{blue}{\tt #1}}}
%\newcommand{\cmd}[1]{{\color{green}{\tt #1}}}
\newcommand{\qtt}[1]{``{\tt #1}"}
\newcommand{\qm}[1]{{\em ``#1"}}

\setlength{\emergencystretch}{3em}

%\documentclass[12pt,a4paper]{article}
\usepackage[hmargin=.47in,vmargin=0.6in,nohead]{geometry}
%\usepackage{fontawesome}
\usepackage{nopageno}
\usepackage{titlesec}
%\usepackage[colorlinks,urlcolor=blue]{hyperref}
%\usepackage{arev}
%\usepackage[T1]{fontenc}
\usepackage[scaled]{berasans}
\renewcommand*\familydefault{\sfdefault}  %% Only if the base font of the document is to be sans serif
\usepackage[T1]{fontenc}

%\titlespacing\section{0pt}{12pt plus 4pt minus 2pt}{0pt plus 2pt minus 2pt}
\titlespacing*{\section}{0pt}{18pt plus 4pt minus 2pt}{6pt plus 2pt minus 2pt}

\begin{document}%
{\begin{center}\Large\bf Accessing EPOCH on GitLab\end{center}}\vspace{5mm}

\noindent
So that people outside of Warwick can access the codes, they are currently
hosted on a publicly accessible GitLab server. This is a git hosting platform
which enables users to access the code, obtain support, report bugs and
participate in community discussions.
This is not as complicated as it sounds so here are the basic instructions.

\section*{Where are the codes?}
To begin using the site, you must first register for an account. In a web
browser, after navigating to the page \url{https://cfsa-pmw.warwick.ac.uk}, you
should be faced with a login screen containing two forms as shown in
Figure~\ref{login}. If you already have an account then you can just type in
your username and password into the first set of boxes, but if not then filling
in the second set will create an account which is authenticated via an
automated email.

\scaledcapimage{images/login_boxes}{login}{Login boxes on the GitLab server}{0.36}

\section*{Accessing the code pages:}
Once logged in to the site, you will arrive at your account ``Dashboard''. To
begin with, this will contain no projects (as shown in Figure~\ref{dashboard1})
and you will need to request access to the {\EPOCH} project by clicking on the
appropriate link. After access to the project is granted, your dashboard screen
will look something like that in Figure~\ref{dashboard2}.

\scaledcapimage{images/dashboard1}{dashboard1}{First time login window}{0.36}

\scaledcapimage{images/dashboard2}{dashboard2}{The dashboard view after being
  granted access to projects}{0.36}

\section*{Projects:}
When logging in to the GitLab web site, you are first presented with the
dashboard page as shown in Figure~\ref{dashboard2}. This page contains a list
of all your current projects on the right-hand side of the page and the first
step is to click on the ``EPOCH/epoch'' link to get to the main {\EPOCH} project
page.

All actions are through the buttons on the left-hand side of the project page.
You can report bugs, request new features, or participate in discussions
under ``Issues''. Documentation and source code tarballs can be found under
``Wiki''.

\section*{Best Practice:}
\begin{itemize}
\item Submit all bugs, questions and feature requests through ``Issues''. Avoid emailing developers directly.
\item Major releases are though the ``Wiki/Downloads'' link as tar files.
\item Learn to use git so you can access the full development software tree. It takes a bit of reading but is
definitely worth the effort. Much better than downloading the .tar.gz file.
\item If other people want a copy of the code encourage them to get a GitLab account and login as above
rather than just giving them your copy. This makes it possible to debug problems should they arise.
\item There are manuals available on the project web pages under the ``Wiki/Downloads'' tab. Read them!
\end{itemize}

\section*{Reporting Errors or Requests:}
All bug reports, feature requests or general questions must now be through
the ``Issues'' tab of the Project pages. If you have a question about the code,
how to use it or general questions these should be posted on the ``Issues''
section and assigned either ``support'' or ``discussion'' labels. Also check
if your question has already been answered. This will stop developers wasting
time answering the same question multiple times. If you want to report a bug
these have the label ``bug'' and feature requests have the label ``suggestion''.

\vspace{10mm}
\noindent Tony Arber\\
Centre for Fusion, Space \& Astrophysics\\
University of Warwick

%\LARGE\faDownloadAlt
\end{document}%
