\documentclass[8pt]{article} \usepackage{url,graphicx,tabularx,array}
\usepackage[margin=.5in,nohead,nofoot]{geometry} \usepackage{color}
\begin{document}

\title{CFD File Format Reference Card} \author{Original author: C.S.Brady\\Most
  recent version: K.Bennett}
\maketitle

\section{Definitions}

\begin{tabular}{c | c | c} Name & FORTRAN & C\\ \hline INTEGER &
  INTEGER(KIND=4) & int\\ INTEGER8 & INTEGER(KIND=8)& long long\\ REAL &
  REAL(KIND=*)& *\\ STRING & CHARACTER & char*\\
\end{tabular}\\\\

The KIND of a floating point number is determined within the
file structure. Any reader must be capable of dealing with both single and
double precision floating point numbers. There are currently no plans to
support arbitrary precision floating point.\\\\

To save space in this document, arrays are represented so that\\\\

REAL(2)\\\\

means an array of length 2 of type REAL. If the number in the brackets is
itself an array then it means expand the inner array into multiple indices
for the outer array. This would mean that REAL(npoints) for a 3D array where
(npoints =[nx,ny,nz]) would expand into REAL(nx,ny,nz)\\\\

Where multidimensional arrays are specified they are laid out in FORTRAN style
on disk, first index is striped over first.\\

\section{File Header}
The start of a CFD file will always look like this.\\

\begin{itemize}
\item STRING(3) :: cfd - Always ``CFD'' in a valid file
\item INTEGER :: header\_offset - The length of the file header. Jumps to start
  of blocks
\item INTEGER :: block\_header\_size - The length in bytes of the block
  header. Once a reader has read all that it knows of a block header, it must
  skip on until it is this many bytes from the start of the block to ensure
  that is has passed the entire header.
\item INTEGER :: cfd\_version - Version number. If this is higher than the
  version of the reader then the reader MUST fail, file WILL almost certainly
  not be readable.
\item INTEGER :: cfd\_revision - Revision number. If this is higher than the
  version of the reader then it should try to continue. Revision changes only
  add data to structures, never remove or move data.
\item INTEGER :: max\_string\_len - All strings in the file are of this
  length. Allocate this much space for all future string variables.
\item INTEGER :: nblocks - The number of blocks in the file. When writing this
  should be zero until the file is written. It is also permissable to increment
  this by one once a given block in the file has been written.
\item INTEGER :: endianness - An integer used to indicate the endianness of the
  file. This is a fixed constant which should never need to be changed.
\item INTEGER :: jobid1 - See below
\item INTEGER :: jobid2 - These two integers uniquely identify the simulation
  which wrote the data. Currently the time in seconds and milliseconds is used
  but this may change.
\item INTEGER :: step - The simulation step number.
\item REAL16 :: time - The simulation time.
\end{itemize}

The first block is written immediately after the end of the file
header. Remember to use header\_offset to jump to the start of the blocks
rather than just assuming that you're in the correct place once you've read the
header.

\section{Block Headers}
All blocks have the same header which is of the following form.\\

\begin{itemize}
\item STRING :: block\_name - The name of the current block
\item STRING :: block\_class - The class name of the current block (used to
  group together similar variables)
\item INTEGER :: block\_type - The type of the block. If a reader can't read
  this type of block then it should use block\_length to skip it.
\item INTEGER8 :: metadata\_length - The length of the metadata at the start of
  a block (nx,ny etc. before the real data) INCLUDING the block header. Once a
  reader has read as much of the metadata as it can it should advance from the
  start of the block to metadata\_length to ensure that it has skipped the
  entire metadata.
\item INTEGER8 :: block\_length - The length of the entire block including the
  block header and metadata length. Once a reader has read a block, it should
  skip a distance of block\_length from the start of the block to ensure that
  it is now in position to read the next block.
\end{itemize}

\subsection{Block Type Constants}
\begin{itemize}
\item c\_type\_scribble = -1 - Deleted block. Code should ignore.
\item c\_type\_additional = 0 - internal block. code should ignore.
\item c\_type\_mesh = 1 - block describing a mesh or grid.
\item c\_type\_mesh\_variable = 2 - block describing a variable on a mesh.
\item c\_type\_snapshot = 3 - block describing the timestate of an output.
\end{itemize}

There are additional types, but they are not essential for an
understanding of the format.

\section{c\_type\_mesh}

All types of mesh are subtypes of c\_type\_mesh

\subsection{Mesh Type Constants}
\begin{itemize}
\item c\_mesh\_cartesian = 0 - A Cartesian mesh. Does not have to be uniformly
  space, but must be a simple cartesian mesh.
\item c\_mesh\_particle = 1 - A particle mesh (fully unstructured unlinked
  mesh). Each particle is an independant point in space and is not linked to
  other particles.
\end{itemize}

It is possible to represent other types of mesh in the file format, but this
has not been done yet.

\subsection{Common Mesh Metadata}

All types of mesh block have the following common metadata.
\begin{itemize}
\item INTEGER :: mesh\_type - The type of the mesh (Cartesian, Particle etc.)
\item INTEGER :: ndims - The dimensionality of the mesh
\item INTEGER :: sof - The size of the floating point numbers used in the
  output dump in bytes. Currently only 4 byte (single precision) and 8 byte
  (double precision) are supported.
\end{itemize}

\subsection{c\_mesh\_cartesian}

The metadata and data for a Cartesian mesh is
\begin{itemize}
\item INTEGER(ndims) :: npoints - The number of grid points in each dimension.
\item REAL(2,ndims) :: minmax - The extents of the grid in each dimension.
\item END OF METADATA
\item REAL(npoints(1)) :: data\_1 - The gridpoints for axis 1.
\item REAL(npoints(2)) :: data\_2 - The gridpoints for axis 2.
\item ...
\item REAL(npoints(ndims)) :: data\_ndims - The gridpoints for axis ndims.
\end{itemize}

\subsection{c\_mesh\_particle}

The metadata and data for a Particle mesh is
\begin{itemize}
\item INTEGER :: particle\_coord\_type - The type of coordinate system for the
  particles. This is currently ignored and the particle positions is assumed
  Cartesian.
\item INTEGER8 :: npart - The number of particles.
\item REAL(2,ndims) :: minmax - The extents of the particle range in each
  dimension
\item END OF METADATA
\item REAL(npart) :: particle\_pos\_1 - The particle coordinates in dimension 1
\item ...
\item REAL(npart) :: particle\_pos\_ndim - The particle coordinates in
  dimension nDim.
\end{itemize}

Note that particle positions are stored in the slightly odd order
of xxxxxxxx...yyyyyyyyyy...zzzzzzzz. This will probably change in later
releases of the CFD format since it has turned out to be an unmitigated pain.

\section{c\_type\_mesh\_variable}

The way that mesh variables are written is very similar to the way that meshs
are written. There are several sanity checks in the format which means that
much of the metadata is written for both the mesh variable and the mesh on
which it resides.

\subsection{Mesh Variable Type Constants}
\begin{itemize}
\item c\_var\_cartesian = 0 - A Cartesian variable. Does not have to be
  uniformly space, but must be a simple cartesian mesh.
\item c\_var\_particle = 1 - A particle variable (fully unstructured unlinked
  mesh). Each particle is an independant point in space and is not linked to
  other particles.
\end{itemize}

\subsection{Common Mesh Variable Metadata}
All types of mesh variable block have the following common metadata.
\begin{itemize}
\item INTEGER :: variable\_type - The type of the variable (Cartesian, Particle
  etc.)
\item INTEGER :: ndims - The dimensionality of the variable
\item INTEGER :: sof - The size of the floating point numbers used in the
  output dump in bytes. Currently only 4 byte (single precision) and 8 byte
  (double precision) are supported.
\end{itemize}

\subsection{c\_var\_cartesian}
\begin{itemize}
\item INTEGER(ndims) :: npoints - The number of points in each
  direction. Checked against the value from the mesh.
\item REAL(2) :: extents - The extents of the data represented by the
  data. Note that this is NOT the same as the extents of the mesh, it is the
  range of the data itself.
\item REAL(ndims) :: stagger - For variables which are staggered relative to
  the underlying grid, this parameter represents the distance that the variable
  is from the gridpoint in multiples of the grid spacing.
\item STRING :: mesh\_name - The name of the mesh that the variable is
  associated
\item STRING :: mesh\_class - The class of the mesh that the variable is
  associated. This must match a mesh which will exist when the file is
  completed in both Name and Class or some readers will not work.
\item END OF METADATA
\item REAL(npoints) :: data - The data is written as a FORTRAN ordered
  multidimensional array.
\end{itemize}

\subsection{c\_var\_particle}
\begin{itemize}
\item INTEGER8 :: npart - The number of particles. Checked against the value
  from the mesh.
\item REAL(2) :: extents - The extents of the data represented by the
  data. Note that this is NOT the same as the extents of the mesh, it is the
  range of the data itself.
\item STRING :: mesh\_name - The name of the mesh that the variable is
  associated
\item STRING :: mesh\_class - The class of the mesh that the variable is
  associated. This must match a mesh which will exist when the file is
  completed in both Name and Class or some readers will not work.
\item END OF METADATA
\item REAL(npart) :: data - The data is written as a 1D array with the same
  ordering of particles as when the mesh is written.
\end{itemize}

\section{Ideal sequence for reading CFD files}
This section details the sequence that an ideal CFD reader should follow to
parse a CFD file.
\begin{itemize}
\item Open the file
\item Read the ``CFD'' string to confirm that this is a CFD file.
\item If the ``CFD'' string isn't found then exit
\item Read the rest of the header info.
\item Check that the version is less than or equal to the version of the
  standard that the reader is written to.
\item If not then exit with an error message about getting a new reader
\item Check that the revision is less than or equal to the revision of the
  standard that the reader is written to.
\item If not then print a warning message, but continue
\item If the nblocks field is zero then exit since the file hasn't been
  finished yet.
\item Set the file pointer to header\_offset to skip the file header (NOTE you
  cannot assume that you are at the end of the file header after reading the
  known fields, since it is valid to extend the file header with new
  information and only increment the revision number)
\item Loop over nblocks
\item Store the current position of the filepointer (block\_start)
\item Read the block header
\item Determine whether the reader can deal with the type of block which is
  described by the header. Also check for user request for this variable etc.
\item If not then the filepointer should be set to block\_start + block\_length
  and then cycle back to the start of the loop.
\item Otherwise set the filepointer to be at block\_start +
  block\_header\_size.
\item Store the filepointer (metadata\_start).
\item Read the block metadata for the correct type of block.
\item Move the filepointer to metadata\_start + metadata\_length.
\item Read the block data. It is valid to skip this step if you only want
  access to the block metadata (the VisIt plugin works in this way when
  populating the VisIt metadata server).
\item Since it is possible for a version change to add new data at the end of
  the data section, move the filepointer to block\_start + block\_length.
\item Check whether the filepointer is at the end of the file. The nblocks
  variable should always be accurate, but you should check.
\item {\textbf End Loop}
\item Close the file.
\end{itemize}

Since there is no requirement on the order in which blocks are written into
the file, it is not possible to do sanity checks until the file has been
parsed once, so a good reader should then do the following checks
\begin{itemize}
\item Check that all the meshes which are requested by mesh variables exist. If
  one of the meshes doesn't exist print a warning message and if the code
  cannot cope without a mesh then drop the variable. The code should NOT fail
  at this point if it is possible to continue.
\item Check that the sizes of variables match the sizes of the associated
  meshes. If not then print a warning message and drop the variable. The code
  should NOT fail at this point if it's possible to continue.
\end{itemize}

\end{document}
